%%%%%%%%%%%%%%%%%%%%%%%%%%%%%%%%%%%%%%%%%%%%%%%%%%%%%%%%%%%%%%%%%%%%%%%%%%%%%%%%%%%%%%%%%%%%%%%
%Plantilla: para la realizaci�n de informes.
%Curso:     Simulaci�n estad�stica.
%Profesor:  Johann A. Ospina.
%%%%%%%%%%%%%%%%%%%%%%%%%%%%%%%%%%%%%%%%%%%%%%%%%%%%%%%%%%%%%%%%%%%%%%%%%%%%%%%%%%%%%%%%%%%%%%%


%Establece el tipo de documento (art�culo), tama�o de letra (10pt) a una columna.
\documentclass[letterpaper,12pt,onecolumn,titlepage]{article} 
 
 
% Cargar paquetes
\usepackage{verbatim}
\usepackage{mathrsfs}
\usepackage{amsmath}
\usepackage{amssymb}
\usepackage{subfigure}
\usepackage{ucs}
\usepackage[latin1]{inputenc}
\usepackage[spanish]{babel}
\usepackage{fontenc}
\usepackage{graphicx}
\usepackage{anysize}
\usepackage{fancyhdr}
\usepackage[comma,authoryear]{natbib}
\usepackage{url} %paquete para definir url
\usepackage{hyperref}  %hipervinculos

%Estilo de la p�gina
\pagestyle{fancy}

%Establecer el margen
\marginsize{2cm}{2cm}{1cm}{1cm}
\setlength{\headheight}{13.1pt}


% Portada
\title{
    \textbf{Laboratorio N.3}\
    ~\\{Introduccion a Los Metodos Estadisticos}   
    ~\\{Estimacion por intervalos y simulacion}}
\author{
    {Diana Carolina Arias Sinisterra Cod. 1528008}
 ~\\{Kevin Steven Garcia Chica Cod. 1533173}
 ~\\{Cesar Andres Saavedra Vanegas Cod. 1628466}}

\date{
     \textbf{Universidad Del Valle}\   
    ~\\{Facultad De Ingenieria}
    ~\\{Estadistica}
    ~\\{Octubre 2017}
    ~\\{}}
 
 
 
\decimalpoint %Poner punto decimal
 
\begin{document}
 
% Se aplica el formato a las p�ginas. Se despliegan: portada e �ndices de materias, figuras y tablas
\renewcommand{\listtablename}{}
\renewcommand{\tablename}{Tabla}
\maketitle
\setcounter{page}{2}
\tableofcontents{}
%\thispagestyle{empty}
%\newpage
%\listoffigures{}
%\listoftables

\thispagestyle{empty}

\newpage
\fancyhead{}
\fancyfoot{}
 
% Encabezado y pie de pagina
\lhead{Introduccion a los Metodos Estadisticos}
\lfoot{Universidad Del Valle}
\rfoot{\thepage}

% Estilo de la bibliograf�a
\bibliographystyle{apalike}
 
% Desarrollo de los contenidos del documento
\pagebreak\section{Situaci\'{o}n 1}
~\\ Se generaron 5000 muestras aleatorias de tama\~{n}o n=10, de una poblaci\'{o}n normal con par\'{a}metros $\mu=5$ y $\sigma=1$ y para cada una de las muestras, se encontr\'{o} un intervalo de confianza para la media y la varianza respectivamente, con una confianza del $95\%$.
~\\ \textbf{Para la media:}
~\\Nos arrojo que el porcentaje de intervalos que atrapan la verdadera media es: $95.22\%$
~\\ Y la longitud promedio o esperada de cada intervalo fue hallada como: $\sum\limits_{i=1}^{5000}\frac{(LS_{i}-LI_{i})}{5000}$ lo que nos arrojo un resultado de 1.2395.

~\\El porcentaje de intervalos que atrapan la verdadera media tiene mucho sentido, ya que, al trabajar con una confianza del $95\%$, estamos diciendo que el $95\%$ de las veces que se repita el experimento (en este caso que se obtenga una muestra distinta de la misma poblaci\'{o}n), el verdadero valor de la media caer\'{a} en el intervalo obtenido. Como obtuvimos un porcentaje del $95.22\%$, nos indica que el $95.22\%$ de las veces que se repiti\'{o} el experimento (remuestrear y obtener el IC para la media), la verdadera media cayo en el intervalo encontrado.

~\\ Con respecto a la longitud promedio, esta nos arroja la amplitud que tiene aproximadamente cada intervalo. En este caso, todos los intervalos tienen exactamente la misma longitud, ya que utilizamos la estimaci\'{o}n cuando $\sigma$ es conocida, y en este tipo de estimaci\'{o}n, la parte que se le suma y se le resta a la media muestral para hallar el intervalo de confianza, no depende de los datos. esa parte es $\pm Z_{1-\frac{\alpha}{2}}\cdot\frac{\sigma}{\sqrt{n}}$. Entonces,en este caso, el 1.2395 que nos arrojo la longitud promedio,nos dice que cada uno de los 5000 intervalos tiene una longitud de 1.2395.

~\\ \textbf{Para la varianza:}
\subsection{Punto a.}
~\\ \textbf{PARA LA MEDIA:}
~\\ Utilizamos la siguiente tabla para ver mas f\'{a}cilmente la comparaci\'{o}n por tama\~{n}os de muestra, de la proporci\'{o}n de intervalos que atrapan la verdadera media y de la longitud promedio de los intervalos.
\begin{center}
\begin{tabular}{|c|c|c|c|c|}
\hline 
\rule[-1ex]{0pt}{2.5ex}  & n=10 & n=30 & n=50 & n=100 \\ 
\hline 
\rule[-1ex]{0pt}{2.5ex} Porcentaje de cubrimiento & 0.9522 & 0.9474 & 0.9508 & 0.9464 \\ 
\hline 
\rule[-1ex]{0pt}{2.5ex} Longitud promedio del intervalo & 1.2395 & 0.7156 & 0.5543 & 0.3919 \\ 
\hline 
\end{tabular} 
\end{center}

~\\ \pagebreak La siguiente figura nos muestra los 100 primeros intervalos con cada tama\~{n}o de muestra (10,20,50,100), ya que si elabor\'{a}bamos la gr\'{a}fica con los 5000 intervalos estimados, no nos daba una visi\'{o}n clara ni comparable de lo que ocurre. 
~\\ \begin{figure}[!h]
    \begin{center}
        \includegraphics[width=10cm]{Figuras/Punto1.png}
        \caption{Gr\'{a}fica comparativa de los 100 primeros intervalos por cada tama\~{n}o de muestra}
        \label{fig:Densidad}
    \end{center}
\end{figure}



~\\\textbf{Representaci\'{o}n gr\'{a}fica del porcentaje de cubrimiento esperado de los intervalos de la media:}
~\\ \begin{figure}[!h]
    \begin{center}
        \includegraphics[width=10cm]{Figuras/Pa1.png}
        \caption{Gr\'{a}fica del porcentaje de cubrimiento de los intervalos para $\mu$ por cada tama\~{n}o de muestra}
        \label{fig:Densidad}
    \end{center}
\end{figure}
~\\ En esta imagen podemos ver que todos los porcentajes de cubrimiento est\'{a}n cerca del $95\%$, lo cual es l\'{o}gico, ya que estamos trabajando con una confianza del $95\%$, lo que nos dice que el $95\%$ de las veces que se repita el proceso (remuestrear y hallar el intervalo de confianza) la verdadera media o la media real, va a caer dentro del intervalo estimado. Ademas podemos ver que el porcentaje de cubrimiento disminuye o aumenta sin depender de el tama\~{n}o de muestra n(10,30,50,100), es decir, no se ve un patr\'{o}n claro de dependencia entre el tama\~{n}o de las muestras y el porcentaje de cubrimiento de los intervalos estimados.


~\\ \textbf{Representaci\'{o}n gr\'{a}fica de la longitud esperada por intervalo para la media:}
~\\ \begin{figure}[!h]
    \begin{center}
        \includegraphics[width=10cm]{Figuras/Pa2.png}
        \caption{Gr\'{a}fica de la longitud esperada de cada intervalo para $\mu$ por cada tama\~{n}o de muestra}
        \label{fig:Densidad}
    \end{center}
\end{figure}
~\\ En la anterior imagen observamos que la longitud mas alta es la de n=10, y no sobrepasa de 1.5, y la mas baja es la de n=100 y es de aproximadamente 0.4. Esto nos muestra que hay una correlaci\'{o}n lineal negativa entre el tama\~{n}o de muestra y la longitud del intervalo, es decir, mientras mayor es el tama\~{n}o de muestra n, menor es la longitud del intervalo. Lo anterior tiene mucho sentido, ya que la parte que se le suma y se le resta a la media de cada muestra, para hallar una estimaci\'{o}n por intervalos para la media real $\mu$ es, $Z_{(1-\alpha)}\cdot\frac{\sigma}{\sqrt{n}}$ y vemos que esta expresi\'{o}n es mas peque\~{n}a cuando el n aumenta, haciendo menor, la amplitud de cada intervalo.

\pagebreak \textbf{PARA LA VARIANZA:}
~\\ Utilizamos la siguiente tabla para ver mas f\'{a}cilmente la comparaci\'{o}n por tama\~{n}os de muestra, de la proporci\'{o}n de intervalos que atrapan la verdadera varianza y de la longitud promedio de los intervalos.
\begin{center}
\begin{tabular}{|c|c|c|c|c|}
\hline 
\rule[-1ex]{0pt}{2.5ex}  & n=10 & n=30 & n=50 & n=100 \\ 
\hline 
\rule[-1ex]{0pt}{2.5ex} Porcentaje de cubrimiento & 0.9524 & 0.9454 & 0.9496 & 0.9502 \\ 
\hline 
\rule[-1ex]{0pt}{2.5ex} Longitud promedio del intervalo & 2.8446 & 1.1712 & 0.8543 & 0.5795 \\ 
\hline 
\end{tabular} 
\end{center}
~\\\textbf{Representaci\'{o}n gr\'{a}fica del porcentaje de cubrimiento esperado de los intervalos de la varianza:}
~\\ \begin{figure}[!h]
    \begin{center}
        \includegraphics[width=10cm]{Figuras/Pb1.png}
        \caption{Gr\'{a}fica del porcentaje de cubrimiento de los intervalos para $\sigma^2$ por cada tama\~{n}o de muestra}
        \label{fig:Densidad}
    \end{center}
\end{figure}
~\\ En esta imagen podemos ver que todos los porcentajes de cubrimiento de los intervalos para la varianza est\'{a}n cerca del $95\%$, lo cual es l\'{o}gico, ya que como se explico en el punto anterior, estamos trabajando con una confianza del $95\%$, lo que nos dice que el $95\%$ de las veces que se repita el proceso (remuestrear y hallar el intervalo de confianza) la verdadera varianza o la varianza real, va a caer dentro del intervalo estimado. Ademas podemos ver que el porcentaje de cubrimiento disminuye o aumenta sin depender de el tama\~{n}o de muestra n(10,30,50,100), es decir, no se ve un patr\'{o}n claro de dependencia entre el tama\~{n}o de las muestras y el porcentaje de cubrimiento de los intervalos estimados para la varianza.


\pagebreak\textbf{Representaci\'{o}n gr\'{a}fica de la longitud esperada por intervalo para la varianza:}
~\\ \begin{figure}[!h]
    \begin{center}
        \includegraphics[width=10cm]{Figuras/Pb2.png}
        \caption{Gr\'{a}fica de la longitud esperada de cada intervalo para $\sigma^2$ por cada tama\~{n}o de muestra}
        \label{fig:Densidad}
    \end{center}
\end{figure}
~\\ En esta imagen, al igual que en la de los intervalos para la media. Vemos que la longitud es mayor para los tama\~{n}os de muestra de n=10 y es menor para los tama\~{n}os de muestra de n=100, lo que nos muestra que entre estas dos variables (tama\~{n}o de muestra y longitud del intervalo) existe una correlaci\'{o}n lineal negativa, es decir, cuando aumenta el tama\~{n}o de muestra, disminuye la longitud de los intervalos estimados.

\pagebreak\subsection{Punto b.}
~\\ Para este punto se simulo una poblaci\'{o}n exponencial con par\'{a}metro $\lambda=\frac{1}{5}$ y de all\'{i}, se extrajeron las respectivas muestras de los distintos tama\~{n}os.
~\\ \textbf{PARA LA MEDIA:}
~\\ Utilizamos la siguiente tabla para ver mas f\'{a}cilmente la comparaci\'{o}n por tama\~{n}os de muestra, de la proporci\'{o}n de intervalos que atrapan la verdadera media y de la longitud promedio de los intervalos para la poblaci\'{o}n exponencial.
\begin{center}
\begin{tabular}{|c|c|c|c|c|}
\hline 
\rule[-1ex]{0pt}{2.5ex}  & n=10 & n=30 & n=50 & n=100 \\ 
\hline 
\rule[-1ex]{0pt}{2.5ex} Porcentaje de cubrimiento & 0 & 0 & 0 & 0 \\ 
\hline 
\rule[-1ex]{0pt}{2.5ex} Longitud promedio del intervalo & 0.2792 & 0.1531 & 0.0869 & 0.0893 \\ 
\hline 
\end{tabular} 
\end{center}
~\\\textbf{Representaci\'{o}n gr\'{a}fica del porcentaje de cubrimiento esperado de los intervalos para la media de una poblaci\'{o}n exponencial:}
~\\ \begin{figure}[!h]
    \begin{center}
        \includegraphics[width=10cm]{Figuras/Pc1.png}
        \caption{Gr\'{a}fica del porcentaje de cubrimiento de los intervalos para $\mu$ por cada tama\~{n}o de muestra, para la poblaci\'{o}n exponencial}
        \label{fig:Densidad}
    \end{center}
\end{figure}
~\\ En esta imagen podemos ver que todos los porcentajes de cubrimiento son del $0\%$, al contrario que en la poblaci\'{o}n normal, en la cual todos los intervalos est\'{a}n cerca del $95\%$. Esto se debe a que la estimaci\'{o}n por intervalos que estamos aplicando, solo se utiliza cuando la poblaci\'{o}n es normal. Es decir, estas formulas no sirven para estimar $\mu$ de una poblaci\'{o}n no normal. Como se ve en la imagen, nos arroja intervalos que no tienen ning\'{u}n sentido con la poblaci\'{o}n que estamos trabajando (en este caso, la exponencial) y que no atrapan el verdadero valor de $\mu$.

\pagebreak\textbf{Representaci\'{o}n gr\'{a}fica de la longitud de los intervalos para la media de una poblaci\'{o}n exponencial:}
~\\ \begin{figure}[!h]
    \begin{center}
        \includegraphics[width=10cm]{Figuras/P1bM.png}
        \caption{Gr\'{a}fica de la longitud de los intervalos para $\mu$ por cada tama\~{n}o de muestra, para la poblaci\'{o}n exponencial}
        \label{fig:Densidad}
    \end{center}
\end{figure}
~\\ En esta imagen podemos ver que las longitudes de los intervalos, son muy peque\~{n}as (la m\'{a}xima es de aproximadamente 0.25), lo cual nos indica que esos intervalos probablemente no tengan una alta confianza, ya que la probabilidad de que no encierren la verdadera media es muy alta. Ademas vemos que a medida que aumentan los tama\~{n}os de muestra, disminuye aun mas la longitud del intervalo, tendiendo casi a 0, volvi\'{e}ndose as\'{i}, un estimador pr\'{a}cticamente puntual, el cual por obvias razones no va a contener el verdadero valor del par\'{a}metro. 

\pagebreak\textbf{PARA LA VARIANZA:}
~\\ Utilizamos la siguiente tabla para ver mas f\'{a}cilmente la comparaci\'{o}n por tama\~{n}os de muestra, de la proporci\'{o}n de intervalos que atrapan la verdadera varianza y de la longitud promedio de los intervalos para la poblaci\'{o}n exponencial.
\begin{center}
\begin{tabular}{|c|c|c|c|c|}
\hline 
\rule[-1ex]{0pt}{2.5ex}  & n=10 & n=30 & n=50 & n=100 \\ 
\hline 
\rule[-1ex]{0pt}{2.5ex} Porcentaje de cubrimiento & 0 & 0 & 0 & 0 \\ 
\hline 
\rule[-1ex]{0pt}{2.5ex} Longitud promedio del intervalo & 0.1089 & 0.0492 & 0.0200 & 0.0293 \\ 
\hline 
\end{tabular} 
\end{center}
~\\\textbf{Representaci\'{o}n gr\'{a}fica del porcentaje de cubrimiento esperado de los intervalos para la varianza de una poblaci\'{o}n exponencial:}
~\\ \begin{figure}[!h]
    \begin{center}
        \includegraphics[width=10cm]{Figuras/Pc2.png}
        \caption{Gr\'{a}fica del porcentaje de cubrimiento de los intervalos para $\sigma^2$ por cada tama\~{n}o de muestra, para la poblaci\'{o}n exponencial}
        \label{fig:Densidad}
    \end{center}
\end{figure}
~\\ En esta imagen podemos ver que todos los porcentajes de cubrimiento son del $0\%$, al contrario que en la poblaci\'{o}n normal, en la cual todos los intervalos est\'{a}n cerca del $95\%$. Esto se debe a que la estimaci\'{o}n por intervalos que estamos aplicando, solo se utiliza cuando la poblaci\'{o}n es normal. Es decir, estas formulas no sirven para estimar $\sigma^2$ de una poblaci\'{o}n no normal. Como se ve en la imagen, nos arroja intervalos que no tienen ning\'{u}n sentido con la poblaci\'{o}n que estamos trabajando (en este caso, la exponencial) y que no atrapan el verdadero valor de $\sigma^2$.

\pagebreak\textbf{Representaci\'{o}n gr\'{a}fica de la longitud de los intervalos para la varianza de una poblaci\'{o}n exponencial:}
~\\ \begin{figure}[!h]
    \begin{center}
        \includegraphics[width=10cm]{Figuras/P1bV.png}
        \caption{Gr\'{a}fica de la longitud de los intervalos para $\sigma^2$ por cada tama\~{n}o de muestra, para la poblaci\'{o}n exponencial}
        \label{fig:Densidad}
    \end{center}
\end{figure}
~\\ En esta gr\'{a}fica, al igual que en la de para la media, vemos que las longitudes son tambi\'{e}n muy peque\~{n}as. y Al igual que en todas las gr\'{a}ficas, la longitud de los intervalos tiene una correlaci\'{o}n lineal negativo con los tama\~{n}os de muestra. Mientras mas grande sea el tama\~{n}o de muestra la longitud tiene a cero (en este caso).

~\\\textbf{CONCLUSI\'{O}N GENERAL:}
~\\ Comparando los resultados para la poblaci\'{o}n normal y la exponencial, podemos ver que los intervalos son eficientes y funcionales solo para los datos que provienen de una poblaci\'{o}n normal, ya que nos da porcentajes de cubrimiento cerca del nivel de confianza asignado. En cambio, los intervalos calculados para los datos que provienen de una poblaci\'{o}n exponencial tienen un porcentaje de cubrimiento de 0 para cada uno de los tama\~{n}os de muestra, y para los dos par\'{a}metros ($\mu$ y $\sigma^2$).
Entonces concluimos que las formulas que tenemos para este tipo de estimaci\'{o}n, solo funcionan para datos que son provenientes de una poblaci\'{o}n normal, ya que si lo aplicamos a datos muestreados de poblaciones diferentes, nos arroja intervalos que no tienen ning\'{u}n sentido, y que no nos sirven en realidad como estimaci\'{o}n de los par\'{a}metros.
\pagebreak\section{Situaci\'{o}n 2}
~\\ Para darnos cuenta que en realidad la mayor\'{i}a de personas aprueba el proyecto de fluoraci\'{o}n del agua, debemos encontrar un intervalo de confianza de la proporci\'{o}n de personas que est\'{a}n a favor, y ver si este esta por encima del 0.5.
~\\ Entonces. Tomando los datos del enunciado tenemos:
~\\ $n=200$, $\hat{P}=\frac{110}{200}=0.55$ (Proporci\'{o}n de personas a favor), $\alpha=0.01$ entonces $1-\alpha=0.99$
~\\ Ahora, para encontrar un intervalo de confianza para la proporci\'{o}n, aplicamos la siguiente formula:

~\\ $IC(P)_{(1-\alpha)\%}=\left[\hat{P}-Z_{1-\frac{\alpha}{2}}\sqrt{\frac{\hat{P}(1-\hat{P}}{n}};\hat{P}+Z_{1-\frac{\alpha}{2}}\sqrt{\frac{\hat{P}(1-\hat{P}}{n}}\right]$

~\\ Reemplazando en la formula, tenemos:

~\\ $IC(P)_{99\%}=\left[0.55-Z_{0.995}\sqrt{\frac{0.55(0.45)}{200}};0.55+Z_{0.995}\sqrt{\frac{0.55(0.45)}{200}}\right]$

~\\ $IC(P)_{99\%}=[0.55-2.5758(0.035178) ; 0.55+2.5758(0.035178)]$

~\\ $IC(P)_{99\%}=[0.4594 ; 0.6406]$

~\\ \textbf{INTERPRETACI\'{O}N:}
 \begin{figure}[!h]
    \begin{center}
        \includegraphics[width=10cm]{Figuras/Grafico1.png}
        \caption{Interpretaci\'{o}n intervalo para proporciones}
        \label{fig:Densidad}
    \end{center}
\end{figure}

~\\ Como asignamos una confianza del $99\%$ a nuestro intervalo, decimos que el $99\%$ de las veces que se repita el experimento, la proporci\'{o}n real de personas que est\'{a}n a favor de que se agregue fluoruro de sodio al agua va a caer en dicho intervalo (entre 0.4594 y 0.6406). Ahora, observando la gr\'{a}fica, podemos ver que en el intervalo esta contenida la probabilidad de 0.5, por lo tanto, la muestra no nos da evidencia para decir que la mayor\'{i}a de personas aprueba el proyecto de fluoraci\'{o}n.

\pagebreak\section{Situaci\'{o}n 3}
\subsection{Punto a.}
 \begin{figure}[!h]
    \begin{center}
        \includegraphics[width=10cm]{Figuras/Grafico2.png}
        \caption{Gr\'{a}fico de puntos y lineas por negocio de unidades vendidas, antes y despu\'{e}s de la publicidad}
        \label{fig:Densidad}
    \end{center}
\end{figure}

~\\ En el an\'{a}lisis exploratorio de datos que se elaboro, encontramos que la mejor gr\'{a}fica para mostrar la efectividad de la campa\~{n}a publicitaria fue la gr\'{a}fica de puntos y lineas. El eje x son cada una de las sucursales o negocios, y el eje y son las unidades vendidas. La linea negra representa las unidades que se vend\'{i}an en dichos negocios antes de la campa\~{n}a publicitaria, y la roja representa las unidades vendidas despu\'{e}s de la campa\~{n}a.
~\\ Podemos observar que en solo dos de los negocios la linea negra esta por encima de la blanca, es decir, se vendieron mas art\'{i}culos antes de la campa\~{n}a que despu\'{e}s de ella; en el resto de negocios (10 negocios)la linea roja esta por encima de la negra, es decir,se vendieron mas art\'{i}culos despu\'{e}s de la campa\~{n}a que antes de ella. Ademas, si miramos detalladamente las diferencias en los negocios en los cuales la linea negra esta por encima de la roja son de tan solo una unidad vendida, en cambio en los negocios en los cuales la linea roja esta por encima de la negra, hay diferencia hasta de 7 unidades vendidas.
~\\ \textbf{Concluimos entonces, que con el an\'{a}lisis exploratorio y con la gr\'{a}fica obtenida, la campa\~{n}a publicitaria si es efectiva.}

\subsection{Punto b.}
~\\ Como las muestras est\'{a}n relacionadas, ya que son tomadas antes y despu\'{e}s de un tratamiento (en este caso la campa\~{n}a publicitaria) a la misma poblaci\'{o}n. Entonces para encontrar este intervalo debemos usar la estimaci\'{o}n para la diferencia de medias para muestras relacionadas.
~\\ La formula que tenemos para este tipo de estimaci\'{o}n es:

~\\ $IC(\mu_{D})_{(1-\alpha)\%}=\bar{d} \pm t_{(1-\frac{\alpha}{2};n-1)}\cdot\frac{Sd}{\sqrt{n}}$

~\\ En este tipo de intervalo de confianza, todo se basa en la diferencia entre cada uno de los valores del antes y despu\'{e}s del tratamiento, por tanto, para mayor comodidad encontramos las diferencias y las a\~{n}adimos a la tabla:
 
~\\ \begin{center}
 \begin{tabular}{|c|c|c|c|c|c|c|c|c|c|c|c|c|}
\hline 
\rule[-1ex]{0pt}{2.5ex} ANTES & 12 & 10 & 15 & 8 & 19 & 14 & 12 & 21 & 16 & 11 & 8 & 15 \\ 
\hline 
\rule[-1ex]{0pt}{2.5ex} DESPU\'{E}S & 11 & 11 & 17 & 9 & 21 & 13 & 16 & 25 & 20 & 18 & 10 & 17 \\ 
\hline 
\rule[-1ex]{0pt}{2.5ex} DIFERENCIA & 1 & -1 & -2 & -1 & -2 & 1 & -4 & -4 & -4 & -7 & -2 & -2 \\ 
\hline 
\end{tabular} 
\end{center}

~\\ Encontramos que: $\bar{d}=\frac{\sum\limits_{i=1}^{12}d_{i}}{12}=\frac{-27}{12}=-2.25$ y $Sd=\sqrt{\frac{\sum\limits_{i=1}^{12}(d_{i}-\bar{d})^2}{11}}=2.2613$

~\\ Ahora, reemplazando en la formula, nos queda:
~\\ $IC(\mu_{D})_{95\%}=[-2.25 \pm t_{(0.975;11)}\cdot \frac{2.2613}{\sqrt{12}}]$
~\\ $IC(\mu_{D})_{95\%}=[-2.25 \pm 2.2009 \cdot \frac{2.2613}{\sqrt{12}}]$
~\\ $IC(\mu_{D})_{95\%}=[-3.6880 ; -0.8119]$

~\\ \textbf{En conclusi\'{o}n, un intervalo de confianza para la diferencia de medias de unidades vendidas durante un mes antes y un mes despu\'{e}s de la campa\~{n}a es (-3.6880 ; -0.8119).}

\pagebreak
\subsection{Punto c.}
~\\ Para interpretar el intervalo obtenido y darnos cuenta si la campa\~{n}a publicitaria es efectiva o no, realizamos la siguiente imagen:
\begin{figure}[!h]
    \begin{center}
        \includegraphics[width=10cm]{Figuras/Grafico3.png}
        \caption{Interpretaci\'{o}n intervalo para diferencia de medias en muestras relacionadas}
        \label{fig:Densidad}
    \end{center}
\end{figure}

~\\ Como asignamos una confianza del $95\%$ a nuestro intervalo, quiere decir que el $95\%$ de las veces que se repita el experimento, la diferencia real de las medias va a caer entre (-3.6880 y -0.8119). Al ver la imagen, observamos que el intervalo obtenido no contiene al cero y ademas esta debajo de el, lo que nos indica que $\mu_2>\mu_1$, y esto nos dice que el promedio de ventas despu\'{e}s de la campa\~{n}a publicitaria es mayor que el promedio de ventas antes de la campa\~{n}a. 

~\\ \textbf{En conclusi\'{o}n, podemos decir que la campa\~{n}a publicitaria es efectiva ya que aumenta el promedio de ventas.}

\pagebreak\section{Situaci\'{o}n 4}
~\\ Dada una muestra de tama\~{n}o n de una poblaci\'{o}n con una proporci\'{o}n p (desconocida) y
la proporci\'{o}n de muestra estimada  $\hat{p}$ (conocida).

\subsection{Punto a.}

~\\ Se construye una funci\'{o}n para calcular el intervalo de confianza para p con un nivel de significancia $\alpha$. La cual nos arroja un vector de dos elementos: el primero es el limite inferior del IC, y el segundo es el limite superior del IC.

~\\ $IC(P)_{(1-\alpha)\%}=\left[\hat{P}-Z_{1-\frac{\alpha}{2}}\sqrt{\frac{\hat{P}(1-\hat{P}}{n}};\hat{P}+Z_{1-\frac{\alpha}{2}}\sqrt{\frac{\hat{P}(1-\hat{P}}{n}}\right]$


\subsection{Punto b.}

~\\Se genera 5000 muestras de tama\~{n}o 40 de una distribuci\'{o}n binomial con una proporci\'{o}n de 0.85 y un intervalo de confi 
anza del $95\%$ de la proporci\'{o}n de la cual arroja los siguientes resultados:


~\\El intervalo de confianza para:
~\\ n=40, $\hat{P}=0.85$, con un nivel de confianza del $95\%$ es de: $[0.8070307 ;0.9929693 ]$
~\\ De las 5000 simulaciones , 4689 contienen la verdadera proporci\'{o}n de poblaci\'{o}n P(0.85), teniendo entonces una cobertura del $93.78\%$

~\\De lo que se puede decir del intervalo es que al tener una longitud angosta lo hace mas preciso pues de esta forma no tendr\'{a} tanta variabilidad lo que conlleva a tener un porcentaje de cobertura bastante alto


\subsection{Punto c.}

~\\El intervalo de confianza para: n=10,20,30,50,100, se podr\'{a}n visualizar los resultados en la siguiente tabla, la cual contiene el tama\~{n}o de n,el intervalo de confianza, la longitud del intervalo, el contador que hace referencia al numero de veces en que de las 5000 simulaciones contienen la verdadera proporci\'{o}n de poblaci\'{o}n P(es decir con p=0.85),por ultimo la tabla contiene el porcentaje de cobertura del intervalo de confianza.


~\\ \begin{center}
 \begin{tabular}{|c|c|c|c|c|}
\hline 
\rule[-1ex]{0pt}{2.5ex} n & Intervalo & Longitud del Intervalo & Contador & Porcentaje \\ 
\hline 
\rule[-1ex]{0pt}{2.5ex} n=10 & [0.7140-1.0859] & 0.3719 & 3917 & 78.34 \\ 
\hline 
\rule[-1ex]{0pt}{2.5ex} n=20 & [0.4991-0.9008] & 0.4017 & 4063 & 81.26 \\ 
\hline 
\rule[-1ex]{0pt}{2.5ex} n=30 & [0.6153-0.9180] & 0.3027 & 4693 & 93.86 \\ 
\hline 
\rule[-1ex]{0pt}{2.5ex} n=50 & [0.6416-0.8783] & 0.2364 & 4703 & 94.06 \\ 
\hline 
\rule[-1ex]{0pt}{2.5ex} n=100 & [0.8163-0.9436] & 0.1273 & 4632 & 92.64 \\ 
\hline 

\end{tabular} 
\end{center}


~\\ Haciendo la variaci\'{o}n en el n se puede concluir que a medida que este se va volviendo m\'{a}s grande el intervalo de confianza se va acotando, es decir, que este se va volviendo m\'{a}s angosto, lo que significa que se hace mas preciso, ya que no hay tanta variabilidad en los intervalos, dando as\'{i} porcentajes de cobertura bastante altos



\pagebreak\section{Situaci\'{o}n 5}
~\\ Como las muestras de las b\'{a}sculas est\'{a}n relacionadas, ya que una de ellas tiene sus medidas certificadas mientras que la otra se deben verificar. Entonces para encontrar un intervalo de confianza   debemos usar la estimaci\'{o}n para la diferencia de medias para muestras relacionadas.
~\\ La formula que tenemos para este tipo de estimaci\'{o}n es:

~\\ $IC(\mu_{1}-\mu_{2} )_{(1-\alpha)\%}=(\bar{x}_{1} - \bar{x}_{2}) \pm t_{(n_{1}+ n_{2} -2 ; 1-\frac{\alpha}{2})}S{p}\sqrt{\frac{1}{n_1}+\frac{1}{n_2}}$ \\


~\\ En este tipo de intervalo de confianza,  se basa en la diferencia entre cada uno de los valores de las bascula en la cual se quiere verificar su estado de calibraci\'{o}n con respecto a la b\'{a}scula que se encuentra certificada, se tienen los siguientes valores:
 
~\\ \begin{center}
 \begin{tabular}{|c|c|c|c|c|c|c|c|c|c|c|}
\hline 
\rule[-1ex]{0pt}{2.5ex} B\'{A}SCULA 1 & 11.23 & 14.36 & 8.33 & 10.5 & 23.42 & 9.15 & 13.47 & 6.47 & 12.4 & 19.38 \\ 
\hline 
\rule[-1ex]{0pt}{2.5ex} B\'{A}SCULA 2 & 11.27 & 14.41 & 8.35 & 10.52 & 23.41 & 9.17 & 13.52 & 6.46 & 12.45 & 19.35 \\ 
\hline 

\end{tabular} 
\end{center}

~\\ Encontramos que: 

~\\$\bar{x}_{1}=12.871$ y $\bar{x}_{2}=12.891$

~\\$S^2{x}_{1}=26.687$ y $S^2{x}_{2}=26.593$\\
~\\$S{x}_{1}=5.166$ y $S{x}_{2}=5.156$\\
~\\ $1-\alpha=0.98$ ,  $\alpha= 0.02$\\
~\\ $Sp=\sqrt{\frac{(n_{1}-1)S_{1}^2+(n_{2}-1)S_{2}^2}{n_{1}+n_{2}-2}}= 5.16$

~\\ Ahora, reemplazando en la formula para hallar el intervalo de confianza, nos queda:
~\\ $IC(\mu_{1}-\mu_{2} )_{(1-\alpha)\%}=( 12.871 - 12.891) \pm t_{(18 ;0.99)}(5.16)\sqrt{\frac{1}{10}+\frac{1}{10}}$ \\
~\\ $IC(\mu_{D})_{98\%}=[-0.02 \pm 2.552 (5.16) \frac{1}{\sqrt{5}}]$
~\\ $IC(\mu_{D})_{98\%}=[-5.91 ; 5.87 ]$

~\\ Podemos ver, que el intervalo obtenido contiene al cero, y de hecho, el cero es casi el centro del intervalo, lo que nos indica que con un $98\%$ la diferencia entre las medias reales de cada una de las basculas puede ser cero o en simbolos $\mu_{B1}=\mu_{B2}$. Por lo cual, concluimos que la bascula 1 (B1) esta en buen estado y bien calibrada, ya que comparandola con la bascula certificada (B2) la diferencia entrelas medias cae en un intervalo que contien al cero.   


\pagebreak\section{Situaci\'{o}n 6}
\subsection{Punto a.}
~\\ Para darnos cuenta si las varianzas se pueden considerar como iguales o diferentes, debemos encontrar un intervalo de confianza para la raz\'{o}n de las varianzas.
~\\ Extrayendo la informaci\'{o}n del enunciado, tenemos:
~\\ $1-\alpha=0.98$ entonces $\alpha=0.02$
~\\ Ahora, la formula para obtener un intervalo de confianza para la raz\'{o}n de varianzas es:

~\\ $IC(\frac{\sigma_1^{2}}{\sigma_2^{2}})_{(1-\alpha)\%}=\left[\frac{S_{1} ^{2}}{S_{2} ^{2}}\cdot F_{(\frac{\alpha}{2},n_{2}-1,n_{1}-1)}  ; \frac{S_{1} ^{2}}{S_{2} ^{2}}\cdot F_{(1-\frac{\alpha}{2},n_{2}-1,n_{1}-1)} \right]$

~\\ Hallamos $S_{1}^2$ y $S_{2}^2$:
~\\ $S_{1}^2=\frac{\sum\limits_{i=1}^{10}(x_{1i}-\bar{x_1})^2}{9}=76875.9550$ y $S_{2}^2=\frac{\sum\limits_{i=1}^{8}(x_{2i}-\bar{x_2})^2}{7}=42382.41037$
~\\ Reemplazando todo en la formula del intervalo, nos queda:

~\\ $IC(\frac{\sigma_1^{2}}{\sigma_2^{2}})_{98\%}=\left[\frac{76875.9550}{42382.41037}\cdot F_{(0.01,7,9)}  ; \frac{76875.9550}{42382.41037}\cdot F_{(0.99,7,9)} \right]$

~\\ $IC(\frac{\sigma_1^{2}}{\sigma_2^{2}})_{98\%}=[1.8138\cdot0.14884  ;  1.8138\cdot5.61287]$

~\\ $IC(\frac{\sigma_1^{2}}{\sigma_2^{2}})_{98\%}=[0.26997 ; 10.18098]$
~\\ Para interpretar mas f\'{a}cilmente el intervalo obtenido, hicimos la siguiente gr\'{a}fica:
\begin{figure}[!h]
    \begin{center}
        \includegraphics[width=10cm]{Figuras/Grafico4.png}
        \caption{Interpretaci\'{o}n intervalo para la raz\'{o}n de varianzas}
        \label{fig:Densidad}
    \end{center}
\end{figure}
~\\ Como asignamos una confianza del $98\%$ quiere decir, que el $98\%$ de las veces que se repita el experimento con las mismas condiciones, la raz\'{o}n de varianzas poblacionales va a estar entre (0.26997 y 10.18098). Si nos fijamos en la imagen, vemos que el intervalo obtenido contiene al 1. Entonces concluimos que no se pueden considerar las varianzas poblacionales como diferentes, ya que la raz\'{o}n de varianzas poblacionales puede ser 1.
\subsection{Punto b.}
~\\ Para recomendar o no el uso del revestimiento como mecanismo complementario, encontraremos un intervalo de confianza para diferencia de medias, y as\'{i} darnos cuenta si la resistencia promedio de las tuber\'{i}as aumento, disminuyo, o se mantuvo igual.
~\\ Como informaci\'{o}n tenemos: $\bar{x_1}=2902.8$ , $\bar{x_2}=3108.125$, $S_{1}^2=76875.9550$ y $S_{2}^2=42382.41037$

~\\ Debemos aplicar la estimaci\'{o}n para diferencia de medias con $\sigma_{1}^2$ y $\sigma_{2}^2$ desconocidas pero iguales, ya que en el punto anterior, mostramos que con una confianza del $98\%$ las varianzas poblacionales pueden ser iguales.
~\\ Sustituyendo los valores en la formula, tenemos:

~\\ $IC(\mu_1 - \mu_2)_{98\%}=(2902.8-3108.125)\pm t_{(0.99;16)}\cdot Sp\sqrt{\frac{1}{10}+\frac{1}{8}}$ 

~\\ Debemos hallar Sp:
~\\ $Sp=\sqrt{\frac{(n_{1}-1)S_{1}^2+(n_{2}-1)S_{2}^2}{n_{1}+n_{2}-2}}=\sqrt{\frac{9\cdot76875,9550+7\cdot42382.41037}{10+8-2}}=\sqrt{61785.02922}=248.56594$

~\\ Ahora, reemplazando todo, nos queda:

~\\ $IC(\mu_1 - \mu_2)_{98\%}=[-205.325 \pm 2.58349 \cdot 248.56594 \cdot 0.47434]$

~\\ $IC(\mu_1 - \mu_2)_{98\%}=[-509.9308 ; 99.2808]$

\pagebreak
~\\ \textbf{INTERPRETACI\'{O}N:}

~\\ Para interpretar mas f\'{a}cilmente el intervalo obtenido, elaboramos la siguiente gr\'{a}fica:
\begin{figure}[!h]
    \begin{center}
        \includegraphics[width=10cm]{Figuras/Grafico5.png}
        \caption{Interpretaci\'{o}n intervalo para diferencia de medias en muestras independientes}
        \label{fig:Densidad}
    \end{center}
\end{figure}
~\\ Como el intervalo encontrado tiene una confianza del $98\%$, decimos que el $98\%$ de las veces que se repita el experimento en las mismas condiciones, la diferencia de medias de resistencia reales va a caer entre -509.9308 y 99.2808. Ademas, si detallamos la imagen, podemos ver que el intervalo obtenido contiene al cero, esto quiere decir que $\mu_1=\mu_2$ o son muy cercanos. Por lo que concluimos que el uso del revestimiento no eleva la resistencia promedio de las tuber\'{i}as, y no es recomendable el uso de este como mecanismo complementario.

\pagebreak\section{Situaci\'{o}n 7}
~\\ Para comparar los tres estimadores propuestos para el CV, se genero una poblaci\'{o}n para cada distribuci\'{o}n (normal, gamma y uniforme) y se obtuvieron 5000 muestras aleatorias para cada uno de los distintos tama\~{n}os de muestra (5,10,20,30,...,100). Las poblaciones sobre las cuales se obtuvieron las muestras fueron definidas de la siguiente manera:
~\\ 1. $Normal(\mu=200,\sigma=\sqrt{400})$
~\\ 2. $Gamma(\alpha=0.5,\beta=100)$
~\\ 3. $Uniforme(a=200,b=600)$

\section*{Distribuciones a Evaluar}
Partiendo de los resultados obtenidos es importante mencionar que el desempe\^{n}o de los estimadores para cada una de las distribuciones propuestas, estimadores que se presentaran de forma gr\'{a}fica para as\'{i} dar una mejor comprensi\'{o}n del comportamiento de los estimadores propuestos para las distribuciones Normal, Gamma y Uniforme con par\'{a}metros iguales para poder que la comparaci\'{o}n de estimadores sea la apropiada.  

\pagebreak
\begin{figure}[!h]
    \begin{center}
        \includegraphics[width=10cm]{Figuras/Rplot.png}
        \caption{Proporci\'{o}n de cobertura y longitud promedio de los intervalos para $\sigma$ de la distribuci\'{o}n normal}
        \label{fig:Densidad}
    \end{center}
\end{figure}

\begin{figure}[!h]
    \begin{center}
        \includegraphics[width=10cm]{Figuras/Rplot01.png}
        \caption{Proporci\'{o}n de cobertura y longitud promedio de los intervalos para $cv$ de la distribuci\'{o}n normal}
        \label{fig:Densidad}
    \end{center}
\end{figure}

\begin{figure}[!h]
    \begin{center}
        \includegraphics[width=10cm]{Figuras/Rplot02.png}
        \caption{Proporci\'{o}n de cobertura y longitud promedio de los intervalos para $\sigma$ de la distribuci\'{o}n gamma}
        \label{fig:Densidad}
    \end{center}
\end{figure}

\begin{figure}[!h]
    \begin{center}
        \includegraphics[width=10cm]{Figuras/Rplot03.png}
        \caption{Proporci\'{o}n de cobertura y longitud promedio de los intervalos para $cv$ de la distribuci\'{o}n gamma}
        \label{fig:Densidad}
    \end{center}
\end{figure}

\begin{figure}[!h]
    \begin{center}
        \includegraphics[width=10cm]{Figuras/Rplot04.png}
        \caption{Proporci\'{o}n de cobertura y longitud promedio de los intervalos para $\sigma$ de la distribuci\'{o}n uniforme}
        \label{fig:Densidad}
    \end{center}
\end{figure}

\begin{figure}[!h]
    \begin{center}
        \includegraphics[width=10cm]{Figuras/Rplot05.png}
        \caption{Proporci\'{o}n de cobertura y longitud promedio de los intervalos para $cv$ de la distribuci\'{o}n uniforme}
        \label{fig:Densidad}
    \end{center}
\end{figure}

\pagebreak
\subsection{Distribuci\'{o}n Normal}
Para la distribu\'{o}n normal se puede observar como el estimador 2 es el que mejor rendimiento ofrece ya que es el que tiene mayor cobertura en la proporci\'{o}n de los datos, y como a su vez es el que mejor longitud de intervalo tiene en comparaci\'{o}n al estimador 1 y 3.

\subsection{Distribuci\'{o}n  Gamma}
Para la Distribuci\'{o}n  Gamma se observa como el estimador 2 es el que mejor resultados puede ofrecer por encontrarse justo dentro del nivel de confianza requerido, a diferencia de los estimadores 1 y 3, donde tenemos que el estimador 1 queda muy por debajo del nivel de confianza planteado por lo cual se descarta por no atrapar los suficientes datos o como sucede en el estimador 3 que queda por encima sobreestimando al parametro.

\subsection{Distribuci\'{o}n  Uniforme}
Para la Distribuci\'{o}n  Uniforme se pude observar como el estimador 2 arroja una longitud de intervalo mas acotado a comparaci\'{o}n de los otros estimadores, de lo que se puede decir que es mas preciso ya que no tienen tanta variabilidad 

\section*{Conclusi\'{o}n}
Para las distribuciones en general, se observa como el estimador 2 es el que tiene mejor cobertura para los intervalos del coeficiente de variaci\'{o}n y de la desviaci\'{o}n est\'{a}ndar puesto que este se encuentra justo en un nivel de confianza optimo para atrapar los datos tanto para el porcentaje de cobertura como para la longitud promedio de intervalo a diferencia de los estimadores 1 y 3 de los cuales se podr\'{i}a catalogar como el de menor rendimiento al estimador 3, puesto que este es el que menor cobertura tiene y el que mayor longitud promedio muestra.  

\bibliography{Bibliografia}
\end{document}